\section{Implementation}

\subsection{Overview}
This section explains the current implementation of the developed tool set, it
is primarily targeted to fulfill the dissertation's requirements. But is also
aiming to be helpful for users wanting to understand the underlying systems and
prepare them for potentially joining the development effort. Abridged code
extracts are used as of their state for thesis submission, while the main
principles will hold, later readers are asked to please consult the actual
source code if any discrepancies arise or reexport the document. First the main libraries are shortly
explained in their relevance to the program, second the largely abstract plugin
architecture is shown, third the main plugins are presented and last the
delivery processes to the end users are described.


All implementation files are contained and delivered inside a single git
repository, which is freely available online:
\url{https://github.com/ksjogo/oxrti}. All following file paths are relative to that repository's root. All future development will
be immediately available there and the current compiled software version is
always fed automatically from it into the hosted version at \url{https://oxrtimaster.azurewebsites.net/api/azurestatic}.

\subsection{Libraries}
\begin{description}
\item[TypeScript:]  The official header line of TypeScript show some points why
  it was picked for this project: ``TypeScript is a typed superset of JavaScript that compiles
  to plain JavaScript. Any browser. Any host. Any OS\@. Open
  source.''\cite{noauthor_typescript_2018} Which fits
  requirements~\ref{req_os},~\ref{req_system}. Whereas plain JavaScript would
  have allowed slightly easier initial on-boarding and maybe easier immediate
  code `hacks', TypeScript will provide better stability in the long run and a
  quite improved developer experience (requirement~\ref{req_dx}) in the long
  run. With the full typed hook system (compare \autoref{sec_hooks}) it ensures
  that a compiled plugin will not have runtime type problems, reducing the
  amount of switching between code editor and the running software. The whole
  project is setup in a way to fully embrace editor tooling, Visual Studio
  Code\cite *{noauthor_visual_nodate} and Emacs\cite*{noauthor_gnu_nodate} are
  the `officially' tested editors of the project. Code is recommended as it will
  support all developer features out of the box. The installation of the
  tslint\cite*{noauthor_tslint_nodate} plugin\cite*{noauthor_tslint_nodate-1} is
  recommended to keep a consistent code style, which is configured within the
  \emph{tslint.json} file. Most importantly TypeScript adds type declarations
  (and inference) to JavaScript, e.g.:
\begin{typescript}
const thing = function (times: number, other: (index: number) => boolean) { ... }
\end{typescript}
would define \emph{thing} as a function, taking a numbers as first argument and
another function (taking a number as first parameter and returning a boolean) as
second argument. The other most used TypeScript features inside the codebase are
Classes\cite*{noauthor_classes_nodate},
Decorators\cite*{noauthor_decorators_nodate} and
Generics\cite*{noauthor_generics_nodate}, which will be discussed at their first
appearance inside the code samples.

\item[React:] The two main points on React's official website are
  ``Declarative'' and ``Component Based''\cite*{noauthor_react_nodate}, which
  is best shown by an extended example from their website, which exemplifies
  multiple patterns found through the oxrti implementation. The most important
  concept is the jump from having a stateful HTML document, which the JavaScript
  code is manipulating directly, e.g.:
  \begin{typescript}
document.getElementById('gsr').innerHTML="<p>You shouldn't do this</p>"
\end{typescript}
Which is diametric to requirements~\ref{req_easy} and~\ref{req_dx} as it would
require developers to manually keep track of all data cross-references (e.g.\
the pan values having to automatically adapt to the current zoom level). A
declarative approach instead allows much better and easier implemented
reactiveness and better performance (requirements~\ref{req_performance}
and~\ref{req_react}) as the necessary changes can be track and components be
updated selectively.

\begin{typescript}
// a class represents a single component
class Timer extends React.Component {
  // the parent component can pass on props to it
  constructor(props) {
    super(props);
    this.state = { seconds: 0 };
  }

  tick() {
    //  the state is updated and the component is automatically rerendered
    this.setState(prevState => ({
      seconds: prevState.seconds + 1
    }));
  }

  // called after the component was created/added to the browser window
  componentDidMount() {
    this.interval = setInterval(() => this.tick(), 1000);
  }

  // called before the component will be deleted/removed from the browser window
  componentWillUnmount() {
    clearInterval(this.interval);
  }

  // the actual rendering code
  // html can be directly embedded into react components
  // {} blocks will be evaluated when the render method is called
  // which will happen any time the props or its internal states updates
  render() {
    return (
      <div>
        Seconds: {this.state.seconds}
      </div>
    );
  }
}

// mountNode is a reference to a DOM Node
// the component will be mounted inside that node
ReactDOM.render(<Timer />, mountNode);
\end{typescript}
In conjunction with mobx and TypeScript no classes are used for React components
though, but instead Stateless Functional Components
(`SFCs'\cite*{ikeuchi_react_2017}). These SFCs are plain functions, only
depending on their passed properties:
\begin{typescript}
function SomeComponent(props: any) {
 return <p>{props.first} {props.first}</p>
}
\end{typescript}
This component could then be used by:
\begin{typescript}
 <SomeComponent first="Hello" second="World"/>
\end{typescript}
This component systems allows the plugins to define some components and then
`link' them into the program via the hook system, which will be explored later.
\item[mobx:] Its main tagline is ``Simple, scalable state
  management''\cite*{noauthor_mobx:_2018}. An introducing overview is shown
  in~\autoref{mobxflow}. Broadly speaking MobX introduces observable objects.
  Instead of mentioned DOM handling or property passing inside React trees,
  components can just retrieve their values from the observable objects and will
  be automatically refreshed if the read values change. This for example makes
  the implementation of the QuickPan plugin extremely easy, as it can just read
  the zoom, pan, etc.\ values of the other plugins and will automatically receive
  all updates without any further manual observation handling.
  \sidefig{mobxflow}{MobX Flow}{Taken from
    Weststrate\cite*{noauthor_mobx:_2018}. Actions in the oxrti context are most
  often initially user actions, which are then calling into plugins to change
  the state. The state is mostly encapsulated on a plugin basis with usage of
  the mobx-state-tree library, which also encapsulates most computed values.
  Reactions are most often the previously discussed React components.}
\item[mobx-state-tree:] ``Central in MST (mobx-state-tree) is the concept of a
  living tree. The tree consists of mutable, but strictly protected
  objects''\cite*{noauthor_mobx-state-tree:_2018} This allows the implementation
  to have one shared state tree which can be used to safely access all data. All
  nodes inside the state tree are MobX observables. A simple tree with plain MST
  would look like this:
\begin{typescript}
// define a model type
const Todo = types
 .model("Todo", {
  // state of every model
  title: types.string,
  done: false
 })
 .actions(self => ({
  //methods bounds to the current model instance
  toggle() {
   self.done = !self.done
  }
 }))
// create a tree root, with a property todos
const Store = types.model("Store", {
    todos: types.array(Todo)
})
\end{typescript}
This syntax was deemed to convoluted, as it is a lot more complex than standard
JavaScript/TypeScript classes, which were introduced by the ES6 version, as shown in the React description above and
thus being in conflict with requirement~\ref{req_easy}.
\item[classy-mst:] There is an option to use a more traditional syntax instead
  though, with the classy-mst library, with
  which the example above becomes\cite*{noauthor_classy-mst:_2018}:
\begin{typescript}
const TodoData = types.model({
	title: types.string,
	done: false

});

class TodoCode extends shim(TodoData) {
	@action
	toggle() {
		this.done = !this.done;
	}
}

const Todo = mst(TodoCode, TodoData, 'Todo');
\end{typescript}
 Weststrate, the original author of MobX initially was sceptic of this
 syntax\cite*{noauthor_alternative_nodate} as it was changing the semantics of
 ES6 classes, as classy-mst's methods will be automatically bound to the
 instance. This boundness is an advantage for this implementation though, as the
 hook configurations can just refer to \code{this.someMethod} instead of \code{this.someMethod.bind(this)}.
The \code{@action} is a decorator, enabling the following method to change the
state/properties of the model, as MST prohibits that by default. Reactions/View
updates will only happen after the outermost action finished executing.
\item[WebGL:] The increasing support of the WebGL stack is the main reason, why
  it is now feasible to implement a full RTI software stack with plain web
  technologies, as it ``enables web content to use an API based on OpenGL ES 2.0
  to perform 3D rendering in an HTML \code{<canvas>} in browsers that support it
  without the use of plug-ins."\cite*{noauthor_webgl_nodate-1}
  \fig{webglcomp}{WebGL compability}{WebGL compability as from the Mozilla
    Developer network\cite*{noauthor_webgl_nodate}.}

  \todoT{gl-react}
\item[gl-react:]\todoT{gl-react}
\item[webpack:]\todoT{webpack}\cite{renaudeau_gl-react_2018}
\item[electron:]\todoT{electron}
\item[misc:]\todoT{misc}\cite{noauthor_mobx:_2018}
\end{description}

\subsection{Plugin architectures}
\todoT{Plugin architectures}

\begin{typescript}
  function murks() : number{}
\end{typescript}


\hypertarget{btf-file-format}{%
\section{BTF File Format}\label{sec_fileformat}}

This section describes the BTF file format. The aim of this file format
is to provide a generic container for BTF data to be specified using a
variety of common formats. Files shall have the \texttt{.btf.zip}
extension.

\hypertarget{file-structure}{%
\subsection{File Structure}\label{file-structure}}

A BTF file is a ZIP file containing the following:
\begin{itemize}
\item A \textbf{manifest}
file in JSON format, named \texttt{manifest.json}. The manifest contains
all information about the BRDF/BSDF model being used, including the
names for the available \textbf{channels} (e.g. \texttt{R}, \texttt{G}
and \texttt{B} for the 3-channel RGB), the names of the necessary
\textbf{coefficients} (e.g.~bi-quadratic coefficients) and the
\textbf{image file format} for each channel.
\item A single folder named
\texttt{data}, with sub-folders having names in 1-to-1 correspondence
with the channels specified in the manifest.
\item Within each channel
folder, greyscale image files having names in 1-to-1 correspondence with
the coefficients specified in the manifest, each in the image file
format specified in the manifest for the corresponding channel. For
example, if one is working with RGB format (3-channels named \texttt{R},
\texttt{G} and \texttt{B}) in the PTM model (five coefficients
\texttt{a2}, \texttt{b2}, \texttt{a1}, \texttt{b1} and \texttt{c},\!
specifying a bi-quadratic) using 16-bit greyscale bitmaps, the file
\texttt{/data/B/a2.bmp} is the texture encoding the \texttt{a2}
coefficient for the blue channel of each point in texture space.
\item The
datafiles are all in reversed scanline order (meaning from bottom to
top), to keep aligned with the original PTM format and allow easier
loading into WebGL.
\end{itemize}

In case of usage with the oxrti viewer, following files can be present in
addition to those mentioned above:
\begin{itemize}
\item A \textbf{Layers} file in JSON format, named \texttt{oxrti\_layers.json}.
  In addition a \texttt{layers} folder should be present, containing the
  textures for the layers.
\item A \textbf{State} file in JSON format, named \texttt{oxrti\_state.json}.

\end{itemize}

\hypertarget{manifest}{%
\subsection{Manifest}\label{manifest}}

The manifest for the BTF file format is a JSON file with root
dictionary. The \texttt{root} element has two mandatory child elements:
one named \texttt{data}, and one named \texttt{name} with the option of
additional child elements (with different names) left open to future
extensions of the format.
\begin{itemize}
\item The \texttt{name} element is a string with a
name of the contained object.
\item The \texttt{data} element has for
entries, named \texttt{width}, \texttt{height}, \texttt{channels} and
\texttt{channel-model}. The \texttt{width} and \texttt{height}
attributes have values in the positive integers describing the
dimensions of the BTDF. The \texttt{channel-model} attribute has value a
non-empty alphanumeric string uniquely identifying the BRDF/BSDF colour
model used by the BTF file (see the Options section below). The \texttt{channels} element has an arbitrary amout of
named \texttt{channel} entries, according to the \texttt{channel-model}.
\item Additionally the \texttt{data} element has one untyped entry named
\texttt{formatExtra}, where format implementation specific data can be
stored.
\item Each \texttt{channel} has an \texttt{coefficents} child
consisting of an arbitrary number of \texttt{coefficent} entries, as
well as one \texttt{coefficient-model} attribute. The
\texttt{coefficient-model} attribute has value a non-empty alphanumeric
string uniquely identifying the BRDF/BSDF approximation model used by
the BTF file (see the Options section below).
\item Each \texttt{coefficient} element has one attribute: \texttt{format}.
The \texttt{format} attribute has value a non-empty alphanumeric string
uniquely identifying the image file format used to store the channel
values (see the Options section below).
\end{itemize}

\hypertarget{textures}{%
\subsection{Textures}\label{textures}}

Each image file \texttt{/data/CHAN/COEFF.EXT} has the same dimensions
specified by the \texttt{width} and \texttt{height} attributes of the
\texttt{data} element in the manifest, and is encoded in the greyscale
image file format specified by the \texttt{format} attribute of the
unique \texttt{coefficient} element with attribute \texttt{name} taking
the value \texttt{COEFF} (the extension \texttt{.EXT} is ignored). The
colour value of a pixel \texttt{(u,v)} in the image is the value for
coefficient \texttt{COEFF} of channel \texttt{CHAN} in the BRDF/BSDF for
point \texttt{(u,v)}, according to the model jointly specified by the
values of the attribute \texttt{model} for element \texttt{channels}
(colour model) and the attribute \texttt{model} for element
\texttt{coefficients} (approximation model).

In case of collated coefficients,
the texture will not be greyscale, but RGB or RGBA and
the coefficients are split into the RGBA channels in order of their appearance
inside the \texttt{coefficient}'s name.

\hypertarget{options}{%
\subsection{Options}\label{options}}

At present, the following values are defined for attribute
\texttt{channel-model} of element \texttt{channels}.
\begin{itemize}
\item \texttt{RGB}: the
3-channel RGB colour model, with channels named \texttt{R}, \texttt{G}
and \texttt{B}. This colour model is currently under implementation.
\item \texttt{LRGB}: the 4-channel LRGB colour model, with channels named
\texttt{L}, \texttt{R}, \texttt{G} and \texttt{B}. This colour model is
currently under implementation.
\item \texttt{SPECTRAL}: the spectral
radiance model, with an arbitrary non-zero number of channels named
either all by wavelength (format \texttt{-\/-\/-nm}, with
\texttt{-\/-\/-} an arbitrary non-zero number) or all by frequency
format \texttt{-\/-\/-Hz}, with \texttt{-\/-\/-} an arbitrary non-zero
number. This colour model is planned for future implementation.
\end{itemize}

At present, the following values are defined for attribute
\texttt{model} of element coefficients, where the ending character
\texttt{*} is to be replaced by an arbitrary number greater than or
equal to 1.
\begin{itemize}
\item \texttt{flat}: flat approximation model (no dependence on
light position). This approximation model is currently under
implementation.
\item \texttt{RTIpoly*}: order-\texttt{*} polynomial
approximation model for RTI (single view-point BRDF). This approximation
model is currently under implementation.
\item \texttt{RTIharmonic*}:
order-\texttt{*} hemispherical harmonic approximation model for RTI
(single view-point BRDF). This approximation model is currently under
implementation.
\item \texttt{BRDFpoly*}: order-\texttt{*} polynomial
approximation model for BRDFs. This approximation model is planned for
future implementation.
\item \texttt{BRDFharmonic*}: order-\texttt{*}
hemispherical harmonic approximation model BRDFs. This approximation
model is planned for future implementation.
\item \texttt{BSDFpoly*}:
order-\texttt{*} polynomial approximation model for BSDFs. This
approximation model is planned for future implementation.
\item \texttt{BSDFharmonic*}: order-\texttt{*} spherical harmonic
approximation model for BSDFs. This approximation model is planned for
future implementation.
\end{itemize}

At present, the following values are defined for attribute
\texttt{format} of elements tagged \texttt{coefficient}, where the
ending character \texttt{*} is the bit-depth, to be replaced by an
allowed positive multiple of 8.
\begin{itemize}
\item \texttt{BMP*}: greyscale BMP file
format with the specified bit-depth (8, 16, 24 or 32). Support for this
format is currently under implementation.
\item \texttt{PNG*}: PNG file
format encoding the specified bit-depth (8, 16, 24, 32, 48 or 64).
Support for this format is currently under implementation. Different PNG
colour options are used to support different bit-depths:
\item \texttt{Grayscale} with 8-bit/channel to encode 8-bit bit-depth.
\item \texttt{Grayscale} with 16-bit/channel to encode 16-bit bit-depth.
\item \texttt{Truecolor} with 8-bit/channel to encode 24-bit bit-depth.
\item \texttt{Truecolor\ and\ alpha} with 8-bit/channel to encode 32-bit
bit-depth.
\item \texttt{Truecolor} with 16-bit/channel to encode 48-bit
bit-depth.
\item \texttt{Truecolor\ and\ alpha} with 16-bit/channel to
encode 64-bit bit-depth.
\end{itemize}

\hypertarget{Layers}{%
  \subsection{Layers}\label{Layers}}
The layer file is a JSON file with a root array, containing a sorted list of layer
information dictionaries, which consist of 3 attributes each:
\begin{itemize}
\item A \texttt{name} JSON string. It is also the prefix of the layer's
  texture inside the \texttt{layers} folder, where the corresponding
  \texttt{name.png} file resides.
\item A \texttt{texture} object with no further attributes, to replaced at
  runtime with the actual texture data.
\item A \texttt{id} JSON string, to be changed each time a layer texture is exported.
\end{itemize}

\hypertarget{State}{%
  \subsection{State}\label{State}}
The state file is a JSON file with oxrti implementation details and is unlikely
to be of further use for other software, it basically stores the state of all
loaded plugins.

\subsection{Loader}


\subsection{State Management}
\todoT{state management}
\todoT{state import/export}
\todoD{redux}
\todoD{mobx actions}

\subsection{Components}
\todoT{single component units}

\subsection{Renderer Stack}
\todoT{base rendering nodes}
\todoD{stacked components}
\todoD{effects}
\subsubsection{Texture Loading}


\subsection{Hooks}\label{sec_hooks}
The hook system allows stable and prioritized interactions between the different
plugins. All available hooks are declared inside the {Hook.tsx} file, which
offers 3 different types of hooks:
\sourceBE{Hook.tsx}
These types are used to first declare single hook types (which will be discussed
within the plugins consuming them) and then construct the whole hook
configuration tree for all plugins:
\begin{typescript}
type HookTypes = {
  ActionBar?: ConfigHook<ActionBar>,
  AfterPluginLoads?: FunctionHook,
  AppView?: ComponentHook,
...
}
\end{typescript}


\subsection{Plugins}
\todoT{Plugins API}

\subsubsection{Base Plugin}
\todoT{Base Plugin}


\subsubsection{BaseTheme Plugin}
\todoT{Basetheme Plugin}

\subsubsection{RedTheme Plugin}

\subsubsection{TabView Plugin}
\source{AppHooksBegin}{AppHooksEnd}{Hook.tsx}
\todoT{TabView Plugin}

\subsubsection{SingleView Plugin}
\todoT{SingleView Plugin}

\subsubsection{Converter Plugin}
\todoT{Converter Plugin}
\subsubsection{PTMConverter Plugin}
\todoT{PTMConverter Plugin}
\subsubsection{Renderer Plugin}
\source{RendererHooksBegin}{RendererHooksEnd}{Hook.tsx}
\todoT{Renderer Plugin}
\todoT{Base Node}
\todoT{WebGL texture packing}
\subsubsection{PTM Renderer Plugin}
\todoT{PTM Renderer Plugin}
\todoT{Dynamic Shaders}
\todoT{RGB vs LRGB}

\subsubsection{Light Control Plugin}
\todoT{Light Control Plugin}
\subsubsection{Rotation Plugin}
\todoT{Rotation Plugin}
\subsubsection{Zoom Plugin}
\todoT{Zoom Plugin}
\subsubsection{QuickPan Plugin}
\todoT{Zoom Plugin}
\subsubsection{Paint Plugin}
\todoT{Zoom Plugin}
\subsubsection{Import Export Plugin }
\todoT{Automatic Import Export}

\subsection{Applications}
\todoT{Other related graphics}
\todoT{Applications}
\subsubsection{Standalone Website}
\todoT{Standalone Website}
\subsubsection{Embeddable}
\todoT{Embeddable}
\subsubsection{Electron}
\todoT{Electron App deliverable}
