\section{Implementation}

\subsection{Overview}
This section explains the current implementation of the developed tool set, it
is primarily targeted to fulfill the dissertation's requirements. But is also
aiming to be helpful for users wanting to understand the underlying systems and
prepare them for potentially joining the development effort. Abridged code
extracts are used as of their state for thesis submission, while the main
principles will hold, later readers are asked to please consult the actual
source code if any discrepancies arise or reexport the document. First the main libraries are shortly
explained in their relevance to the program, second the largely abstract plugin
architecture is shown, third the main plugins are presented and last the
delivery processes to the end users are described.


All implementation files are contained and delivered inside a single git
repository, which is freely available online:
\url{https://github.com/ksjogo/oxrti}. All following file paths are relative to that repository's root. All future development will
be immediately available there and the current compiled software version is
always fed automatically from it into the hosted version at \url{https://oxrtimaster.azurewebsites.net/api/azurestatic}.

\subsection{Libraries}
\subsubsection*{TypeScript}
 The official header line of TypeScript show some points why
  it was picked for this project: ``TypeScript is a typed superset of JavaScript that compiles
  to plain JavaScript. Any browser. Any host. Any OS\@. Open
  source.''\cite{noauthor_typescript_2018} Which fits
  requirements~\ref{req_os},~\ref{req_system}. Whereas plain JavaScript would
  have allowed slightly easier initial on-boarding and maybe easier immediate
  code `hacks', TypeScript will provide better stability in the long run and a
  quite improved developer experience (requirement~\ref{req_dx}) in the long
  run. With the full typed hook system (compare \autoref{sec_hooks}) it ensures
  that a compiled plugin will not have runtime type problems, reducing the
  amount of switching between code editor and the running software. The whole
  project is setup in a way to fully embrace editor tooling, Visual Studio
  Code\cite *{noauthor_visual_nodate} and Emacs\cite*{noauthor_gnu_nodate} are
  the `officially' tested editors of the project. Code is recommended as it will
  support all developer features out of the box. The installation of the
  tslint\cite*{noauthor_tslint_nodate} plugin\cite*{noauthor_tslint_nodate-1} is
  recommended to keep a consistent code style, which is configured within the
  \emph{tslint.json} file. Most importantly TypeScript adds type declarations
  (and inference) to JavaScript, e.g.:
\begin{typescript}
const thing = function (times: number, other: (index: number) => boolean) { ... }
\end{typescript}
would define \emph{thing} as a function, taking a numbers as first argument and
another function (taking a number as first parameter and returning a boolean) as
second argument. The other most used TypeScript features inside the codebase are
Classes\cite*{noauthor_classes_nodate},
Decorators\cite*{noauthor_decorators_nodate} and
Generics\cite*{noauthor_generics_nodate}, which will be discussed at their first
appearance inside the code samples.
\subsubsection*{React}
The two main points on React's official website are
  ``Declarative'' and ``Component Based''\cite*{noauthor_react_nodate}, which
  is best shown by an extended example from their website, which exemplifies
  multiple patterns found through the oxrti implementation. The most important
  concept is the jump from having a stateful HTML document, which the JavaScript
  code is manipulating directly, e.g.:
  \begin{typescript}
document.getElementById('gsr').innerHTML="<p>You shouldn't do this</p>"
\end{typescript}
Which is diametric to requirements~\ref{req_easy} and~\ref{req_dx} as it would
require developers to manually keep track of all data cross-references (e.g.\
the pan values having to automatically adapt to the current zoom level). A
declarative approach instead allows much better and easier implemented
reactiveness and better performance (requirements~\ref{req_performance}
and~\ref{req_react}) as the necessary changes can be track and components be
updated selectively.

\begin{typescript}
// a class represents a single component
class Timer extends React.Component {
  // the parent component can pass on props to it
  constructor(props) {
    super(props);
    this.state = { seconds: 0 };
  }

  tick() {
    //  the state is updated and the component is automatically rerendered
    this.setState(prevState => ({
      seconds: prevState.seconds + 1
    }));
  }

  // called after the component was created/added to the browser window
  componentDidMount() {
    this.interval = setInterval(() => this.tick(), 1000);
  }

  // called before the component will be deleted/removed from the browser window
  componentWillUnmount() {
    clearInterval(this.interval);
  }

  // the actual rendering code
  // html can be directly embedded into react components
  // {} blocks will be evaluated when the render method is called
  // which will happen any time the props or its internal states updates
  render() {
    return (
      <div>
        Seconds: {this.state.seconds}
      </div>
    );
  }
}

// mountNode is a reference to a DOM Node
// the component will be mounted inside that node
ReactDOM.render(<Timer />, mountNode);
\end{typescript}
In conjunction with mobx and TypeScript no classes are used for React components
though, but instead Stateless Functional Components
(`SFCs'\cite*{ikeuchi_react_2017}). These SFCs are plain functions, only
depending on their passed properties:
\begin{typescript}
function SomeComponent(props: any) {
 return <p>{props.first} {props.first}</p>
}
\end{typescript}
This component could then be used by:
\begin{typescript}
 <SomeComponent first="Hello" second="World"/>
\end{typescript}
This component systems allows the plugins to define some components and then
`link' them into the program via the hook system, which will be explored later.
\subsubsection*{MobX}
Its main tagline is ``Simple, scalable state
  management''\cite*{noauthor_mobx:_2018}. An introducing overview is shown
  in~\autoref{mobxflow}. Broadly speaking MobX introduces observable objects.
  Instead of mentioned DOM handling or property passing inside React trees,
  components can just retrieve their values from the observable objects and will
  be automatically refreshed if the read values change. This for example makes
  the implementation of the QuickPan plugin extremely easy, as it can just read
  the zoom, pan, etc.\ values of the other plugins and will automatically receive
  all updates without any further manual observation handling.
  \sidefig{mobxflow}{MobX Flow}{Taken from
    Weststrate\cite*{noauthor_mobx:_2018}. Actions in the oxrti context are most
  often initially user actions, which are then calling into plugins to change
  the state. The state is mostly encapsulated on a plugin basis with usage of
  the mobx-state-tree library, which also encapsulates most computed values.
  Reactions are most often the previously discussed React components.}
\subsubsection*{mobx-state-tree}
``Central in MST (mobx-state-tree) is the concept of a
  living tree. The tree consists of mutable, but strictly protected
  objects''\cite*{noauthor_mobx-state-tree:_2018} This allows the implementation
  to have one shared state tree which can be used to safely access all data. All
  nodes inside the state tree are MobX observables. A simple tree with plain MST
  would look like this:
\begin{typescript}
// define a model type
const Todo = types
 .model("Todo", {
  // state of every model
  title: types.string,
  done: false
 })
 .actions(self => ({
  //methods bounds to the current model instance
  toggle() {
   self.done = !self.done
  }
 }))
// create a tree root, with a property todos
const Store = types.model("Store", {
    todos: types.array(Todo)
})
\end{typescript}
This syntax was deemed to convoluted, as it is a lot more complex than standard
JavaScript/TypeScript classes, which were introduced by the ES6 version, as shown in the React description above and
thus being in conflict with requirement~\ref{req_easy}.
\subsubsection*{classy-mst}
There is an option to use a more traditional syntax instead
  though, with the classy-mst library, with
  which the example above becomes\cite*{noauthor_classy-mst:_2018}:
\begin{typescript}
const TodoData = types.model({
	title: types.string,
	done: false

});

class TodoCode extends shim(TodoData) {
	@action
	toggle() {
		this.done = !this.done;
	}
}

const Todo = mst(TodoCode, TodoData, 'Todo');
\end{typescript}
 Weststrate, the original author of MobX initially was sceptic of this
 syntax\cite*{noauthor_alternative_nodate} as it was changing the semantics of
 ES6 classes, as classy-mst's methods will be automatically bound to the
 instance. This boundness is an advantage for this implementation though, as the
 hook configurations can just refer to \code{this.someMethod} instead of \code{this.someMethod.bind(this)}.
The \code{@action} is a decorator, enabling the following method to change the
state/properties of the model, as MST prohibits that by default. Reactions/View
updates will only happen after the outermost action finished executing.
\subsubsection*{WebGL}
The increasing support of the WebGL stack is the main reason, why
  it is now feasible to implement a full RTI software stack with plain web
  technologies, as it ``enables web content to use an API based on OpenGL ES 2.0
  to perform 3D rendering in an HTML \code{<canvas>} in browsers that support it
  without the use of plug-ins."\cite*{noauthor_webgl_nodate-1} OpenGL ES 2.0
  likeness means that (most importantly) shaders are supported, allowing the
  implementation to be split up into multiple shaders with single
  responsibilities, for details refer to~\autoref{sec_rendererstack}. While preserving compatibility and
  requirement~\ref{req_system} WebGL 2 support is sadly not widespread enough to fully rely on yet
  (compare~\autoref{webglcomp}), as it is currently estimated at 50\% of all devices\cite*{noauthor_webgl_nodate-2}.
  \fig{webglcomp}{WebGL compability}{WebGL compability as from the Mozilla
    Developer network\cite*{noauthor_webgl_nodate}.}  Potential improvements
  when WebGL 2 is more widely supported or in conditional plugins are discussed
  in~\autoref{sec_futureGL}. One notable limitation of WebGL is
  \code{MAX_TEXTURE_IMAGE_UNITS}, the maximum amount of bound textures inside a
  single shader, which in most implementations is
  16\cite*{noauthor_webgl_nodate-3}, whereas the standard OpenGL implementations
  are likely to have a limit of 32. This is influencing the BTF file format, as
  for example in the PTM RGB use case a total of 18 coefficients exist, which
  now need to be bundled up somehow into maximum 16 textures, if the
  calculations should be done inside a single shader. It is also limiting the
  amount of layers of the Paint plugin, as these also consist of bound textures.
  Apart from the shaders, which are written in the OpenGL ES
  Shading Language\cite*{noauthor_webgl_nodate-4} and the the texture loader (\autoref{seq_textureloader}), no direct
  WebGL code is necessary nor used anywhere inside the implementation, as the
  gl-react library is abstracting it neatly for use from the MobX/React environment.

  \subsubsection*{gl-react}
 ``Implement complex effects by composing React
  components.''\cite*{renaudeau_gl-react_2018} is the main use of the gl-react
  library. A minimal component, adapted from the gl-react-cookbook looks
  like\cite*{renaudeau_gl-react_2018}:
\begin{typescript}
const shaders = Shaders.create({
  helloGL: {
    frag: GLSL`
     precision highp float;
     varying vec2 uv;
     void main() {
      gl_FragColor = vec4(uv.x, uv.y, 0.5, 1.0);
   }`
 }
});

export default class Example extends Component {
  render() {
    return (
      <Surface width={300} height={300}>
        <Node shader={shaders.helloGL} />
      </Surface>
    );
  }
}
\end{typescript}
  Which would result in a display like~\autoref{glreactcookbook}.
  \fig{glreactcookbook}{gl-react-cookbook example}{RGBA texture, with R and G
    according to their respective u or v texture coordinate. From
    \cite*{noauthor_gl-react_nodate}.} gl-react's is not a 3D engine, so no
  objects are to be created or scene graph managed, instead the oxrti
  implementation can concentrate on solely providing the necessary shaders.\
  gl-react's default node size is taken from the parent surface size. The
  surface size will be dependent on the user running the program and his browser
  windows, which makes it undesirable as details would be lost, if the BTF
  provided more detail, so the processing Node sizes are usually set to the BTF
  resolution or higher.
  \subsubsection*{Webpack}
  Webpack is used to bundle the implementation into single files as it is ``a bundler for javascript and friends. Packs many modules into a few bundled
  assets.''\cite*{noauthor_webpack/webpack:_nodate} A more detailed discussion
  on the targets is in~\autoref{sec_apps}. Broadly speaking Webpack loads the
  source code inside the \emph{src} directory according to the loaders defined
  inside the \emph{webpack.config.js} file, analyses their dependencies and then
  bundles them together. This makes it possible to have a dependency tree
  spanning 26184 packages from npm, but still providing a single bundled
  application file only 1.5 megabyte large (data as of \today). It also allows
  the dynamic plugin structure by bundling the plugins into a dynamic `context' from
  which single plugins can be loaded at runtime.
  \subsubsection*{Electron}
  Electron is used to ``build cross-platform desktop apps with JavaScript, HTML,
  and CSS''\cite*{noauthor_electron:_2018} While theoretically not necessary to
  fulfill most requirements, as the implementation is compatible with all modern
  web browsers, providing an additional standalone executable provides some advantages:
  \begin{itemize}
\item It is possible to add a more traditional menu-based interface, which the
  browser version could not support.
  \item Stable development environment, as electron-devtools-installer is used
    to provide relevant extensions (React devtools, MobX devtools) by default
    and the hot reloading is reliably tested, which together form a good
    developer experience (requirement~\ref{req_dx})
  \item It allows to preserve the software in a usable, contained state, not
    relying on the user also having a compatible web browser in the future.
\item It allows future development to more directly access resources of the host
  machine, e.g.\ the normal OpenGL stack could be used for calculating the
  coefficients, as it is less resource constrained compared to the WebGL stack.
  \end{itemize}
  \subsubsection*{MaterialUI}
  MaterialUI is succinctly described by ``React components that implement
  Google's Material Design.''\cite*{noauthor_material-ui:_2018}. MaterialUI's
  component are used throughout the app for styling the components, making the
  use of custom CSS largely unnecessary apart from minor positioning fixes. For
  example the Zoom component is defined as:
\begin{typescript}
// Card, CardContent, Tooltip and Button are all components provided by MaterialUI.
// this refers to the Zoom Plugin's controller which
// the content will be automatically refreshed if the refered values change
const Zoom = Component(function ZoomSlider (props) {
    return <Card style={{ width: '100%' }} >
        <CardContent>
            <Tooltip title='Reset'>
                <Button onClick={this.resetZoom} style={{ marginLeft: '-8px' }}>Zoom</Button>
            </Tooltip>
            <Tooltip title={this.scale}>
                <Slider value={this.scale} onChange={this.onSlider} min={0.01} max={30} />
            </Tooltip>
            <Tooltip title='Reset'>
                <Button onClick={this.resetPan} style={{ marginLeft: '-11px' }}>Pan</Button>
            </Tooltip>
            <Tooltip title={this.panX}>
                <Slider value={this.panX} onChange={this.onSliderX} min={-1 * this.scale} max={1 * this.scale} />
            </Tooltip>
            <Tooltip title={this.panY}>
                <Slider value={this.panY} onChange={this.onSliderY} min={-1 * this.scale} max={1 * this.scale} />
            </Tooltip>
        </CardContent>
    </Card>
})
\end{typescript}
It would result in a display like~\autoref{zoomcomponent}.
\fig{zoomcomponent}{Zoom Component}{Zoom Component, the user is dragging the
  zoom slider currently, the mouse pointer is not depicted.}
 \subsubsection*{Misc}
 Further libraries of note are:
\begin{description}
\item[electron-webpack\cite*{noauthor_electron-userland/electron-webpack:_nodate}] is providing the bridging between Webpack and
   Electron, its config is expanded by the \emph{webpack.renderer.additions.js}
   and \emph{webpack.renderer.shared.js} files.
\item[pngjs\cite*{stolwijk_pngjs:_2018}] is providing in-browser bitwise png
  manipulation, required in the converter.
\item[jszip\cite*{noauthor_jszip_nodate}] is providing in-browser zip file
    manipulations, which are fundamental for the BTF fileformat.
\end{description}


\subsection{Overall Architecture}
The implementation is making a distinction between plugin and non-plugin files.
The amount of non-plugin files was aimed for to be as low as possible, as they
are inflexible in all output configurations and will have slightly different
behaviour while developing in regard to reloads. The files not contained in
plugins are the following:
\begin{itemize}
\item \emph{AppState.tsx}, the mobx-state-tree root node representing the whole
  application state in its leafs, detailed in~\autoref{sec_state}.
 \item \emph{BTFFile.tsx}, containing the fileformat implementation and utility
   functions, described in~\autoref{sec_btffile}.
\item \emph{Hook.tsx} and \emph{HookManager.tsx} which provide the whole dynamic
  interaction system between the different plugins, shown
  in~\autoref{sec_hooks}.
\item \emph{Loader.tsx}, \emph{electron/index.tsx}, \emph{renderer/index.tsx}
  and \emph{web/index.tsx}, providing the loading functionality. The Electron
  application has two entry points, one for the main process, which is
  \emph{electron/index.tsx} and one for the in-browser content, which is
  \emph{renderer/index.tsx}. The in-browser one and the plain browser entry point
  \emph{web/index.tsx} both call the \emph{Loader.tsx} to initialise the state
  management and mount the root React component, so the user can interact
  finally. The Loader also handles hot-reloading, it will receive the changed
  source code from Webpack and update the plugins accordingly.
 \item \emph{Plugin.tsx} defining the base class for a plugins, further
   explained in~\autoref{sec_plugins}.
\item \emph{types.d.ts} is providing the custom ambient type declarations for
  software dependencies, which are not providing TypeScript types on their own.
  In case new dependencies are added, they are likely to require and addition there.
\item \emph{Util.tsx} providing general helper functions, largely related to
  some math functions for texture coordinate handling.
\item \emph{loaders/glslify-loader/index.js} is the custom Webpack loader for
  \emph{.glsl} files, allowing e.g.\ the \code{import zoomShader from
    './zoom.glsl'} statement and setting up Webpack to contain the shader source
  in the final bundle.
\item \emph{loaders/oxrtidatatex/OxrtiDataTextureLoader.tsx} is providing direct
  texture loading from in-memory BTF files, discussed in~\autoref{sec_textureloading}.
\end{itemize}

\subsection{BTF FIle}\label{sec_btffile}
The full standalone BTF file format specification can be found inside~\autoref{sec_fileformat}.
The implementation in \emph{BTFFile.tsx} is an in-memory implementation of that
file with following interface, it is mainly a `dumb' data container.
\desc{beginBTF}{endBTF}{BTFFile}
Notable is the \code{PreDownload} hook which is called before the
\code{generateZip} function is called, to allow the ImpExp plugin and the Paint
plugin to fill the BTF with their respective data. This hook is called by the
AppState though, as the BTFFile implementation is fully standalone (apart from
the jszip dependency), in case other software wishes to re-use the implementation.

\subsection{Texture Loader}\label{seq_textureloader}
The bridge between the loaded BTF files and the WebGL contexts needs to be
closed with custom code, as gl-react was not supporting the load of PNG textures
from memory. An abridged version of the code follows, as it is a good example of
the Promise pattern used in some following parts.
\begin{typescript}
loadTexture(config: TexForRender) {
    // keep track of the amount of currently loading textures
    appState.textureIsLoading()
    // shortcut for the WebGL reference
    let gl = this.gl
    // access the raw data buffer from the texture config
    let data = config.data
    // create a Promise
    // it is basically chaining callbacks together
    // and then asynchronously executing each step
    let promise =
        // first create an ImageBitmap object
        // we need to flipY as the textures are orientated naturally inside the BTF file
        // but WebGL is expecting them bottom row first
        createImageBitmap(data, { imageOrientation: 'flipY' })
        // the catch step is called if the previous part failed
        .catch((reason) => {
            // some browers do not like loading a lot of big textures parrelly
            // and will garbage collect them in between and thus fail
            // as a fallback the limiter function is used to limit concurrency of that part to 1
            // this still allow max currency for other environments
            return limiter(() => createImageBitmap(data))
        })
        // the then part is called when the previous step succedded
        .then(img => {
            // create and bind a new WebGL texture
            let texture = gl.createTexture()
            gl.bindTexture(gl.TEXTURE_2D, texture)
            let type: number
            // map the type from the BTF file to a WebGL texture type
            switch (config.format) {
                case 'PNG8':
                    type = gl.LUMINANCE
                    break
                // ...
                case 'PNG32':
                    type = gl.RGBA
                    break
            }
            // load the imageBitmap into the texture
            gl.texImage2D(gl.TEXTURE_2D, 0, type, type, gl.UNSIGNED_BYTE, img)
            gl.texParameteri(gl.TEXTURE_2D, gl.TEXTURE_MIN_FILTER, gl.NEAREST)
            gl.texParameteri(gl.TEXTURE_2D, gl.TEXTURE_MAG_FILTER, gl.NEAREST)
            // finished and return
            appState.textureLoaded()
            return { texture, width: img.width, height: img.height }
        })
        .catch((reason) => {
            // a null texture will be empty
            alert('Texture failed to load' + reason)
            appState.textureLoaded()
            return { texture: null, width: config.width, height: config.height }
        })
   // the calling code will have its own catch/then logic
   return promise
}
\end{typescript}


\subsection{State Management}\label{sec_state}
\todoT{state management}
\todoT{state import/export}
\todoD{redux}
\todoD{mobx actions}


\subsection{Renderer Stack}\label{sec_rendererstack}
\todoT{base rendering nodes}
\todoD{stacked components}
\todoD{effects}
\subsubsection{Texture Loading}\label{sec_textureloading}
\todoT{Texture Loading}


\subsection{Hooks}\label{sec_hooks}
The hook system allows stable and prioritized interactions between the different
plugins. All available hooks are declared inside the {Hook.tsx} file, which
offers 3 different types of hooks:
\sourceBE{Hook.tsx}
These types are used to first declare single hook types (which will be discussed
within the plugins consuming them) and then construct the whole hook
configuration tree for all plugins:
\begin{typescript}
type HookTypes = {
  ActionBar?: ConfigHook<ActionBar>,
  AfterPluginLoads?: FunctionHook,
  AppView?: ComponentHook,
...
}
\end{typescript}


\subsection{Plugins}\label{sec_plugins}
\todoT{Plugins API}

\subsubsection{Base Plugin}
\todoT{Base Plugin}


\subsubsection{BaseTheme Plugin}
\todoT{Basetheme Plugin}

\subsubsection{RedTheme Plugin}

\subsubsection{TabView Plugin}
\source{AppHooksBegin}{AppHooksEnd}{Hook.tsx}
\todoT{TabView Plugin}

\subsubsection{SingleView Plugin}
\todoT{SingleView Plugin}

\subsubsection{Converter Plugin}
\todoT{Converter Plugin}
\subsubsection{PTMConverter Plugin}
\todoT{PTMConverter Plugin}
\subsubsection{Renderer Plugin}
\source{RendererHooksBegin}{RendererHooksEnd}{Hook.tsx}
\todoT{Renderer Plugin}
\todoT{Base Node}
\todoT{WebGL texture packing}
\subsubsection{PTM Renderer Plugin}
\todoT{PTM Renderer Plugin}
\todoT{Dynamic Shaders}
\todoT{RGB vs LRGB}

\subsubsection{Light Control Plugin}
\todoT{Light Control Plugin}
\subsubsection{Rotation Plugin}
\todoT{Rotation Plugin}
\subsubsection{Zoom Plugin}
\todoT{Zoom Plugin}
\subsubsection{QuickPan Plugin}
\todoT{Zoom Plugin}
\subsubsection{Paint Plugin}
\todoT{Zoom Plugin}
\subsubsection{Import Export Plugin }
\todoT{Automatic Import Export}

\subsection{Applications}\label{sec_apps}
\todoT{Other related graphics}
\todoT{Applications}
\subsubsection{Standalone Website}
\todoT{Standalone Website}
\subsubsection{Embeddable}
\todoT{Embeddable}
\subsubsection{Electron}
\todoT{Electron App deliverable}
