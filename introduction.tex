\section{Introduction}
Reflectance Transformation Imaging (RTI) was  invented by Tom Malzbender and Dan
Gelb, research scientists at Hewlett-Packard Labs. It was originally termed
Polynomial Texture Mapping (PTM). It is a method of computational photography
with great potential for classical archaeology. RTI images (RTIs) are created from
multiple digital photographs of an object. A fixed camera position is used in
conjunction with a a movable light source, or multiple immovable light sources, hitting the object from different positions. Different shadows and highlights result from
each light position.

An RTI processing software takes these images and calculates the object's
surface per pixel, essentially creating a 2D photograph with embedded
reflectance information.

This file can then be viewed with the help of an RTI viewing software, allowing
the object to be studied remotely via the user's computer, instead of needing
physical access to the object. The software is also able to reveal enhanced or
previously unobservable details, e.g.\ colour, shape, markings or depth of
carved lines, which the naked eye could not pick up.

Ancient historians studying material objects benefit from RTI because this
software shows how objects were created and subsequently changed. For example
captured RTIs of the wooden remains of Roman wax tablets can reveal how they
have been inscribed and reinscribed. For statues, particularly Roman imperial
portraits RTI can provide evidence for deliberate re-carving of a condemned
emperor into his successor.

For video examples and diagrams, the Cultural Heritage Imaging Institute provides a great overview\cite*{noauthor_cultural_nodate}.

Other applications of RTI are video games, where RTI can provide self-shadowing
for objects, which normal maps could not, though more modern techniques largely
replaced RTI in this context.

They are becoming increasingly relevant in a physical based rendering (PBR) context,
where RTIs can provide better physical information for objects.

This dissertation is primarily focusing on the application of RTI for ancient
historians, but makes provisions for relevance in a PBR context in the future.

\emph{oxrti} in this document is referring to this project's implementation of an
RTI viewer.