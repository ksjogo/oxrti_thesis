\section{Requirements and Design}
\subsection{Requirements}
In preparation for the project the author had multiple informal conversations
with consenting RTI users from different Oxford departments in relation to their
current usage of RTI viewing software and wishes for a modernized software. One
common point was the wish for a unification of the used software, most
researchers had to export screenshots from the RTI software, to then add
annotations and drawings in their graphical editing software. But in their
graphical editing software all the advantages of an RTI image were then lost.
Based on that and the discussed related work following requirements were arrived at.

The first overall goal is the main focus on web technologies. The new viewer
should be available on as many platforms as possible and allow for easy
distribution and usage, as well as a comforting developer experience. Based on
that the importance for a web supported fileformat came next. No browser is supporting
RTI or PTM files natively,  so some conversion needs to take place. The
requirement of an additional conversion software would be diametrical to the
`everything web based' goal though, so the browsers need to be able to also
handle the conversion and then support some kind of export. The easiest way to
do so was to develop a newer file format with following features:

\begin{enumerate}
\item Support for embedding/consuming the PTM\cite*{library_of_congress_polynomial_2018}  fileformat.
\item Future support for embedding/consuming the RTI\cite*{library_of_congress_reflectance_2018} fileformat.
\item Extended metadata support.
\item Support for high resolutions.
\item Support for higher bitdepths per pixel than the 8 of PTM/RTI for future
  HDR applications.
\item Easy exchange between multiple researchers and computers.
\end{enumerate}

For the viewer following features were deemed essential:
\begin{enumerate}[resume]
\item Compatible with all major operating systems and/or web browsers.\label{req_system}
\item Light Controls to change the position of the virtual light source.
\item Multiple rendering modes apart from plain LRGB or RGB.
\item Quick navigation functionality.
\item Annotations to allow for further textual information relating to a part of
  the viewer object.
\item Paintable layers to further clarify scripture.
\end{enumerate}

In addition these non-functional requirements were arrived at:
\begin{enumerate}[resume]
\item  Free Software, the implementation should be available for everyone to
  change and distribute. No email-walling, instead embracing jump-in
  interactions between multiple parties. \label{req_os}
\item Suitable of new developers: for scientists for research purposes,
  or students for educational purposes. \label{req_easy}
\item Plugin support to allow independent additions or modifications to the core software.
\item Good developer experience. \label{req_dx}
\item Adequate performance, at least keeping up with current
  implementations. \label{req_performance}
\item Easy installation for
  researchers. \label{req_install}
\item Fast responses to user interactions. \label{req_react}
\item Reasonable file sizes for instant transfer/viewing.
\item Preservable software and BTF files.\label{req_preserve}
\end{enumerate}

\subsection{Fileformat}
With
Though the LoC deems the PTM format ``''relatively transparent. It can be viewed
in plain text editors. [\ldots] A very simple program would present these
numbers in a human-readable
form.''\cite*{library_of_congress_polynomial_2018} the author disagrees with
this judgment.

\subsection{Architecture}

\subsection{State-Driven}
\todoT{State-Driven}
\subsection{Plugins}
\todoT{Plugins}
\subsection{Rendering Stack}
\todoT{Rendering Stack}

