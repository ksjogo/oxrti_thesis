\section{Discussion}
This section discusses the achieved results and proposes future avenues of research.

\subsection{Novelties}
Though some functionality of current RTI viewers could not be replicated during
this project (mainly \.RTI file format support and, many novel things were
achieved:
\begin{itemize}
\item Free Software first, the source code\cite*{noauthor_oxrti:_2018}, the
  file format specification and accompanying
  documentation\cite*{noauthor_oxrti_thesis_2018} are freely available online from
  first day on.
\item Full plugin orientation making it the most flexible and modifiable
  implementation. The development speed picked up considerably feature wise the
  longer the project went, as most hooks were defined and functions could be
  stitched together to achieve fast turnarounds.
\item Full web compability without local/native dependencies for preprocessing,
  enabling a speedy and comfortable workflow, with no requirements of dropping
  into a command line anywhere.
\item Most transparent and flexible file format for RTI, while being smaller and
  faster to load.
\item Paint plugin, making round trips with graphical editing software
  unnecessary.
\item Flexible shader based architecture for improved performance.
\end{itemize}

\subsection{Future Work}\label{sec_future}
The future work can be split into two parts. Improvements of the current system,
including better performance and bug fixes, and further extensions with new
functionality. The following improvements are currently indicated (in order of importance):
\begin{itemize}
\item Support of \.RTI files. Requiring the addition of a new parser and some
  new rendering base nodes.
\item Support for some rendering modes like specular enhancement and others as
  presented by Palma et al.\cite*{palma_dynamic_2010}.
\end{itemize}

For new features three areas are interesting:

Technical features:
\begin{description}
\item[Compression:] Support for the newly (June 2018) developed techniques by Ponchio et al\@.
  to compress and enhance RTI data for web
  applications\cite*{ponchio_compact_2018}
\item[Animation support:] The state-driven architecture would easily support some
  kind of tweening for light positions, zoom, etc. The image export could be
  extended to support video output.
\item[3D Engines:] Integration of some full blown 3D engine to support virtual scenes, with
  the base nodes providing underlying textures for some objects. Would require a
  reworking of the caching behaviour
\item[HDR images:] The BTF file format can easily support higher
  resolution and higher bit-depth pictures.
\item[WebGL 2] integration with features like texture arrays could simplify some
  calculations. Hold up by WebGL2 compability of common browser though.
\item Rendering tree instead of rendering stack. The biggest performance problem
  currently is the GPU's fill rate. A more flexible tree would require fewer
  drawn pixels per user input, resulting in better performance.
\item[Shader collation:] Currently there is no easy exchange between different
  shader's code. Internal support/use of glslify\cite*{noauthor_glslify/glslify_2018} could help with that and
  potentially allow the dynamic construction of shaders at runtime, purely depending on
  the \emph{manifest.json} and not on handwritten support for all formats.
\end{description}

Being a web application some collaborative features come to mind, if the
addition of a server is considered:
\begin{description}
\item[Online object registry:] Allowing captured objects to be registered under some
  hash to uniquely identify them. Potential hosting of the corresponding data
  file. Standard features like registration, login, privacy settings for objects.
\item[Shared items]: Bookmarks, Notes, Layers, etc.\ could all be exchanged
  online and potentially automatically synced between machines. The
  mobx-state-tree in the background is automatically generating patch files and
  all these objects have unique ids, making realtime exchange quite feasible.
\item[Machine Learning:] Having a comprehensive database of crowd sourced
  labelled data - e.g.\
  labels for single pictograms of captured cuneiforms - could be the base for
  some machine learning on the RTI data. Also specific light configurations
  could be rated in their usefulness for analysing specific parts, feeding into
  an auto lightning button.
\end{description}

The success of a shader based viewer indicates that also the RTI building part
could be ready for some improvements through new oxrti plugins.
\begin{description}
\item[Shader based:] A new builder could use the GPU to parallelize the RTI
  construction process, which currently are single-threaded and CPU bound. A
  modern data set likely consist of more than 64 differently lighted images,
  which would not be transferable at the same time to the (WebGL controlled) GPU directly, but
  two options could be considered: Implementing the building part inside the
  Electron app and its local process, which has full access to a machine's GPU
  without WebGL limitations, but still controllable by the plugin architecture
  through inter-process-communication. Alternatively the `stack of images'
  could instead be split along the z-axis, as each texel only requires all its
  values for all light positions and no values of neighbouring texels, allowing
  for good parallelization.
\item[Configuration based:] The current RTIBuilder (also provided by
  CHI\cite*{noauthor_cultural_nodate-4}) does an light angle estimation per
  captured image. With RTI domes these angles are fixed though could just be
  given to the software for better accuracy.
\item[Video based:] The capture and transfer of a lot of single images, which
  are nearly the same (as the light will only change slightly between each
  picture) could be accelerated by instead capturing a video. The WebGL support
  for video textures could be an entry point.
\end{description}

\subsection{Community Outreach and Onboarding}
The most important next step though is community outreach and onboarding as the
nicest features only become useful if they are used. Outreach will happen in two
different communities. The implementation core is providing a novel and modern
infrastructure for any kind of plugin based software, and as such will be taken
by the author to the JavaScript and TypeScript community. But more important is
the RTI community. Initial outreach on social media platforms generated interest
within the Classics community; the next steps are a more formal introduction of
the software at the CHI forums\cite*{noauthor_chiforums:_nodate}, as they are
frequented by many RTI users and a publication targeted at researchers in the
field of archaeology and/or conservation.

\section{Conclusion}
The project succeeded in developing a modern, exploratory, more `industrialized', web-based
RTI viewer software with improved performance and new functionality; and in
outlining future potential for novelties in applications around RTI\@. Whilst
the author will ensure continued support for the software, the future
plurality of developers and users of this software is of primary importance.