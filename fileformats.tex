\hypertarget{btf-file-format}{%
\subsection{BTF File Format}\label{sec_btf-file-format}}

This section describes the BTF file format. The aim of this file format
is to provide a generic container for BTF data to be specified using a
variety of common formats. Files shall have the \texttt{.btf.zip}
extension.

\hypertarget{file-structure}{%
\subsubsection{File Structure}\label{file-structure}}

A BTF file is a ZIP file containing the following:
\begin{itemize}
\item A \textbf{manifest}
file in JSON format, named \texttt{manifest.json}. The manifest contains
all information about the BRDF/BSDF model being used, including the
names for the available \textbf{channels} (e.g. \texttt{R}, \texttt{G}
and \texttt{B} for the 3-channel RGB), the names of the necessary
\textbf{coefficients} (e.g.~bi-quadratic coefficients) and the
\textbf{image file format} for each channel.
\item A single folder named
\texttt{data}, with sub-folders having names in 1-to-1 correspondence
with the channels specified in the manifest.
\item Within each channel
folder, greyscale image files having names in 1-to-1 correspondence with
the coefficients specified in the manifest, each in the image file
format specified in the manifest for the corresponding channel. For
example, if one is working with RGB format (3-channels named \texttt{R},
\texttt{G} and \texttt{B}) in the PTM model (five coefficients
\texttt{a2}, \texttt{b2}, \texttt{a1}, \texttt{b1} and \texttt{c},\!
specifying a bi-quadratic) using 16-bit greyscale bitmaps, the file
\texttt{/data/B/a2.bmp} is the texture encoding the \texttt{a2}
coefficient for the blue channel of each point in texture space.
\item The
datafiles are all in reversed scanline order (meaning from bottom to
top), to keep aligned with the original PTM format and allow easier
loading into WebGL.
\end{itemize}

In case of usage with the oxrti viewer, following files can be present in
addition to those mentioned above:

\hypertarget{manifest}{%
\subsubsection{Manifest}\label{manifest}}

The manifest for the BTF file format is a JSON file with root
dictionary. The \texttt{root} element has two mandatory child elements:
one named \texttt{data}, and one named \texttt{name} with the option of
additional child elements (with different names) left open to future
extensions of the format.
\begin{itemize}
\item The \texttt{name} element is a string with a
name of the contained object.
\item The \texttt{data} element has for
entries, named \texttt{width}, \texttt{height}, \texttt{channels} and
\texttt{channel-model}. The \texttt{width} and \texttt{height}
attributes have values in the positive integers describing the
dimensions of the BTDF. The \texttt{channel-model} attribute has value a
non-empty alphanumeric string uniquely identifying the BRDF/BSDF colour
model used by the BTF file (see \protect\hyperlink{Options}{Options}
section below). The \texttt{channels} element has an arbitrary amout of
named \texttt{channel} entries, according to the \texttt{channel-model}.
\item Additionally the \texttt{data} element has one untyped entry named
\texttt{formatExtra}, where format implementation specific data can be
stored.
\item Each \texttt{channel} has an \texttt{coefficents} child
consisting of an arbitrary number of \texttt{coefficent} entries, as
well as one \texttt{coefficient-model} attribute. The
\texttt{coefficient-model} attribute has value a non-empty alphanumeric
string uniquely identifying the BRDF/BSDF approximation model used by
the BTF file (see \protect\hyperlink{Options}{Options} section below).
\item Each \texttt{coefficient} element has one attribute: \texttt{format}.
The \texttt{format} attribute has value a non-empty alphanumeric string
uniquely identifying the image file format used to store the channel
values (see \protect\hyperlink{Options}{Options} section below).
\end{itemize}

\hypertarget{textures}{%
\subsubsection{Textures}\label{textures}}

Each image file \texttt{/data/CHAN/COEFF.EXT} has the same dimensions
specified by the \texttt{width} and \texttt{height} attributes of the
\texttt{data} element in the manifest, and is encoded in the greyscale
image file format specified by the \texttt{format} attribute of the
unique \texttt{coefficient} element with attribute \texttt{name} taking
the value \texttt{COEFF} (the extension \texttt{.EXT} is ignored). The
colour value of a pixel \texttt{(u,v)} in the image is the value for
coefficient \texttt{COEFF} of channel \texttt{CHAN} in the BRDF/BSDF for
point \texttt{(u,v)}, according to the model jointly specified by the
values of the attribute \texttt{model} for element \texttt{channels}
(colour model) and the attribute \texttt{model} for element
\texttt{coefficients} (approximation model).

\hypertarget{options}{%
\subsubsection{Options}\label{options}}

At present, the following values are defined for attribute
\texttt{channel-model} of element \texttt{channels}.
\begin{itemize}
\item \texttt{RGB}: the
3-channel RGB colour model, with channels named \texttt{R}, \texttt{G}
and \texttt{B}. This colour model is currently under implementation.
\item \texttt{LRGB}: the 4-channel LRGB colour model, with channels named
\texttt{L}, \texttt{R}, \texttt{G} and \texttt{B}. This colour model is
currently under implementation.
\item \texttt{SPECTRAL}: the spectral
radiance model, with an arbitrary non-zero number of channels named
either all by wavelength (format \texttt{-\/-\/-nm}, with
\texttt{-\/-\/-} an arbitrary non-zero number) or all by frequency
format \texttt{-\/-\/-Hz}, with \texttt{-\/-\/-} an arbitrary non-zero
number. This colour model is planned for future implementation.
\end{itemize}

At present, the following values are defined for attribute
\texttt{model} of element coefficients, where the ending character
\texttt{*} is to be replaced by an arbitrary number greater than or
equal to 1.
\begin{itemize}
\item \texttt{flat}: flat approximation model (no dependence on
light position). This approximation model is currently under
implementation.
\item \texttt{RTIpoly*}: order-\texttt{*} polynomial
approximation model for RTI (single view-point BRDF). This approximation
model is currently under implementation.
\item \texttt{RTIharmonic*}:
order-\texttt{*} hemispherical harmonic approximation model for RTI
(single view-point BRDF). This approximation model is currently under
implementation.
\item \texttt{BRDFpoly*}: order-\texttt{*} polynomial
approximation model for BRDFs. This approximation model is planned for
future implementation.
\item \texttt{BRDFharmonic*}: order-\texttt{*}
hemispherical harmonic approximation model BRDFs. This approximation
model is planned for future implementation.
\item \texttt{BSDFpoly*}:
order-\texttt{*} polynomial approximation model for BSDFs. This
approximation model is planned for future implementation.
\item \texttt{BSDFharmonic*}: order-\texttt{*} spherical harmonic
approximation model for BSDFs. This approximation model is planned for
future implementation.
\end{itemize}

At present, the following values are defined for attribute
\texttt{format} of elements tagged \texttt{coefficient}, where the
ending character \texttt{*} is the bit-depth, to be replaced by an
allowed positive multiple of 8.
\begin{itemize}
\item \texttt{BMP*}: greyscale BMP file
format with the specified bit-depth (8, 16, 24 or 32). Support for this
format is currently under implementation.
\item \texttt{PNG*}: PNG file
format encoding the specified bit-depth (8, 16, 24, 32, 48 or 64).
Support for this format is currently under implementation. Different PNG
colour options are used to support different bit-depths:
\item \texttt{Grayscale} with 8-bit/channel to encode 8-bit bit-depth.
\item \texttt{Grayscale} with 16-bit/channel to encode 16-bit bit-depth.
\item \texttt{Truecolor} with 8-bit/channel to encode 24-bit bit-depth.
\item \texttt{Truecolor\ and\ alpha} with 8-bit/channel to encode 32-bit
bit-depth.
\item \texttt{Truecolor} with 16-bit/channel to encode 48-bit
bit-depth.
\item \texttt{Truecolor\ and\ alpha} with 16-bit/channel to
encode 64-bit bit-depth.
\end{itemize}
