\section{Results}
This sections summarises the results of the project and adds some quantitative
analysis of the implementation. The feature set is fully reflected inside the
implementation section (or the table of contents), so no additional
summarization is done here.

\subsubsection{Performance}
The currently most used RTI viewer is the Cultural Heritage Imaging
one's\cite*{noauthor_cultural_nodate-1}, so the performance comparisons are done
against that implementation. The data is shown
in~\autoref{table_fileformat_results}. The BTF files are about 50\% the size of
the ptm files, with increasing compression benefits the larger the ptm file is.
The same pattern is evident in the load time as~\autoref{loadtimes} shows. The
bigger the files are, the more profitable is the shader implementation.
\fig{loadtimes}{Load time comparison}{Load time comparison.}

\begin{table}[H]
\begin{tabular}{|c |  c c c c c c|}
 \hline
 File & Pixel & Conv & Load & Load RTI & ptm size & btf size\\
  \hline
  \emph{AK1A\_LRGB} & 3008x2000 & 4906 & 1713 & 11800 & 54.1 & 30  \\
  \emph{Coin\_LRGB} & 7360x4912 & 24674 & 8224 & 59200 & 325.4 & 136  \\
  \emph{Mummy\_RGB} & 583x1000 & 1769 & 627 & 5200 & 10.5 & 5.6  \\
  \emph{tablet2\_LRGB} & 512x512 & 732 & 462 & 1600 & 2.4 & 1.2  \\
  \emph{Warrior\_RGB} & 2299x3200 & 10590 & 2184 & 25500 & 132.4 & 50.2  \\
 \hline
\end{tabular}
\caption{Performance Comparison}
\emph{File} names refer to the files distributed as \cite*{goslar_oxrti_data:_2018}.
\emph{Pixel} are the dimensions as width x height. \emph{Conv} is the time taken
by the implementation to convert the ptm file to the btf format. \emph{Load} is
the time taken by the oxrti implementation from receiving the btf file to do a complete render. \emph{Load
  RTI} is the approximate time taken by the RTIViewer from opening the file to
finishing the loading (time stopped by screenrecording, as no performance
measurements are provided, time for processing of mipmaps and normal maps
excluded, as the oxrti implementation is not doing these steps, time to full
first render would be even longer). \emph{ptm size} is the file size in
megabytes. \emph{btf size} is the file size of a plain BTF file (no extra oxrti
state and/or layers).
\label{table_fileformat_results}
\end{table}

\subsection{Accuracy}
The next thing to verify is the accuracy, is the implemented rendering actually
the `right' imagine. To quantify that, the Root Mean Square Error as presented
by Happa and Gogosio\cite*{gogioso_pbr:_2017} was picked, as it the compared
files have
rendering
\begin{table}[H]
\centering
\begin{tabular}{|c |  c c c | c|}
 \hline
 File & RMSE(R) & RMSE(G) & RMSE(B) & SUM  \\
  \hline
  \emph{tablet2\_LRGB} & 0.649 & 0.677 & 0.674  & 2.000 \\
  \emph{Warrior\_RGB} &0.627 & 0.612 & 0.610 & 1.849  \\
 \hline
\end{tabular}
\caption{Table to test captions and labels}
\label{table_Accuracy}
\end{table}

\subsection{Testing}
\todoT{Testing}
\todoT{Shader Interpolation}
\todoT{Image comparison}

\subsection{Rollouts and Deployments}
\todoT{Rollout}
\todoT{Non-Tech deployment}o